\documentclass[11pt,a4paper]{article}

\usepackage{german}
\usepackage[utf8]{inputenc}
\usepackage{multirow}
\usepackage{amsmath}

\title{Übung 5: Aufgabe 2}
\date{24. Oktober 2014}
\author{Barney Golaszewski}

\begin{document}
\maketitle


\section{Das ist der erste Abschnitt}
Bla bla bla bla bla bla bla bla bla bla bla bla bla bla bla bla bla bla bla bla bla bla bla bla bla bla bla bla bla bla bla bla bla bla bla bla bla bla bla bla! \cite{der1955einfuhrung}

\section{Tabelle}
Bla bla bla bla bla bla bla bla bla bla bla!

\begin{table}[h]
\centering
\begin{tabular}{r|c|c|l}
 & Maximalpunktzahl & Punkte erhalten & \% \\
\hline
Aufgabe 1 & 10 & 5 & 50 \\
Aufgabe 2 & 20 & 5 & 25 \\
Aufgabe 3 & 6 & 6 & 100 \\
\end{tabular}
\caption{Punktetabelle \cite{hilbert1928grundlagen}}
\end{table}

\section{Formeln}

\subsection{Pythagoras}
Der Satz des Pythagoras errechnet sich wie folgt: $a^2+b^2=c^2$. Daraus können wir die Länge der Hypothenuse wie folgt berechnen: $c=\sqrt{a^2+b^2}$ \cite{varma2013effect}

\subsection{Summen}
Wir können auch die Formel für eine Summe angeben:
\begin{equation}
s=\sum\limits_{i=1}^ni=\frac{n*(n+1)}{2}
\end{equation}

\bibliography{bibfile}
\bibliographystyle{plain}


\end{document}
